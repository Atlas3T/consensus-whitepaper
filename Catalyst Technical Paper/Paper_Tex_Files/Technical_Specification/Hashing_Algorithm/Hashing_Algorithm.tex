Hashing generally refers to algorithms used to obfuscate data by generating a summary of the data, or hash, in such a way that the original data can not be restored using the hash, \textit{i.e.} the hashing function is a one-way function, while the hash can be used to prove knowledge or ownership of the original data. A hashing function generates a pseudo-random string of fixed length from a data of arbitrary length. The hashing algorithm is said to be collision-resistant when the probability to generate two hash from two different data is negligible. Furthermore a hashing algorithm has the property that two similar data will lead to very different hashes, that is to say a collection of hashes does not allow an entity unaware of the original data to acquire knowledge about the data. \\ 

The hashing algorithm used on Catalyst ledger is 256Blake2b (or simply Blake-2b) which produces a 256-bits string and is known to be amongst the fastest hashing algorithms and particularly suitable for mobile applications~\cite{blake}. Throughout the document, the hashing function is referenced by the symbol $\mathcal{H}$.

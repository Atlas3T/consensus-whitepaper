On the Catalyst ledger, all the transaction entries are signed to authorise the transfer of tokens which means that all the participants in the transaction need to sign their respective entry for a transaction to be considered complete. When signing an entry $E_i$ a participant needs to prove ownership of the account $A_i$ referred in said transaction entry. Said otherwise, the user needs to prove knowledge of the private key $k_i$ paired the public key $Q_i$ from which the account address $A_i$ is derived. A verifier (or producer in Catalyst jargon) can verify the validity of the signature given the public key $P_i$ specified in an entry $E_i$.\\

Signatures for transactions on the Catalyst network are formed in a highly similar way regardless of whether the asset transfer embedded in said transaction is confidential or non-confidential. The signature scheme describes in this section therefore applies to both transaction types unless explicitly stated. \\

In blockchains such as Bitcoin and Ethereum, transaction inputs are signed using ECDSA scheme, where the public key is recovered from the signature and used to retrieve the account or UTXO address, thus ensuring that the rightful owner of the tokens is authorised to spend these. The use of a second temporary, often called ephemeral, public/private key pair in the signature adds a layer of protection against malicious attempt to retrieve the private key of a user when signing multiple transactions spending tokens from the same address.\\
 
 Public key recovery is however incompatible with batch validation, \textit{i.e.} it is not possible to recover a set of public keys from an aggregated signature on multiple transaction inputs. As a result the choice of ECDSA-based scheme for Catalyst transaction would not be optimal as a transaction contain a minimum of two entries. A Schnorr-based signature scheme is preferred to enable user to jointly produce a signature using their private keys. A solution recently proposed by Y. Seurin \textit{et al}~\cite{schnorr} also accounts for protection against key-rogue attacks, preventing key malleability to create validate signatures on transaction without knowing the users' private key. We propose a Schnorr-based signature scheme inspired from this recently published work. \\

We define the transaction core message $m$ as a set of $n$ entries $\{E_i\}_{i=1,..,n}$  and additional information $X$ mentioned in the previous section (see table~\ref{tab:TrSt}):
\begin{center}
$m = \{E_i\}_{i=1,..,n} + X$ 
\end{center}

The participant $U_i$ responsible for $E_i$ creates the following challenge:
\begin{equation} 
e_i = \mathcal{H}(m \mid\mid \tilde{Q})\mathcal{H}(L \mid Q_i)
\end{equation} 

Where:

\begin{itemize}
	\item$\mathcal{H}$ is a hashing function
	\item$m$ is the transaction core message 
	\item$\mid\mid$ denotes concatenation between strings
	\item$\tilde{Q}$ is the aggregated public key such that:
\begin{center}
$\tilde{Q}=\mathcal{H}(L \mid\mid Q_1)\cdot Q_1 + .. + \mathcal{H}(L \mid\mid Q_n)\cdot Q_n$
\end{center}
	\item$L$ is the hash of all the public keys used in the transaction expressed as $L=\mathcal{H}(Q_1~||~..~||~Q_n)$
\end{itemize}

$U_i$ then creates the following partial signature:  

\begin{equation} 
s_{i} = r_i + e_i\cdot k_i
\end{equation}

Where $r_i$ is a pseudo-random number chosen by $U_i$ and kept secret. Recall that each entry $E_i$ in a confidential transaction includes a PC obfuscating the amount $v_i^{t}$ defined by: $C_i^{t} = v_i ^{t} H + b_i^{t} G$. For confidential transaction we define $r_i = b_i^t + d_i$ where $d_i$ is a pseudo-random number chosen by $U_i$ and kept secret.
The partial signature $s_i$ and $R_i = r_iG$ are exchanged amongst the transaction participants. Each participant in the transaction $U_k$ ($k\neq i$) can compute:

\begin{equation} 
R_i' = s_{i}\cdot G - e_i\cdot P_i
\end{equation}

where $P_i = k_i\cdot G$ and verify that $R_i' = R_i$, proving the validity of the partial signature. \\

The last participant to receive the full set of partial signatures builds the transaction signature. The signature of non-confidential transaction is composed of the pair:

\begin{equation} 
T= (\underbrace{s_{1}+..+s_{n}}_{s}, \underbrace{R_{1}+..+R_{n}}_{R})
\end{equation}

At the verification phase, a producer can check that the total signature is as follows:

\begin{enumerate}
\item compute the list of $n$ challenges $\{e_i'\}_{i=1,..,n}$ given the public keys embedded in the transaction entries $\{E_i\}_{i=1,..,n}$ included in $m$
\item compute the quantity $R' = s \cdot G - \sum_{i=1}^{n}e_i'P_i$
\item verify that $R'=R$
\end{enumerate}

If so, the signature $T = (s,R)$ is valid. The signature is composed of a 32-byte integer and a 32-byte EC point, leading to a compact 64-byte signature for the entire transaction. \\

The signature of confidential transaction is defined differently. Using the public keys $\{D_i = d_iG\}_{i=1,..,n}$, it is composed of the pair:

\begin{equation} 
T= (\underbrace{s_{1}+..+s_{n}}_{s}, \underbrace{D_{1}+..+D_{n}}_{D})
\end{equation}

The verifier can compute:

\begin{center}
$R' = \sum_{i=1}^{n}C_i^t - v^fH + D$
\end{center}

If $R' = s \cdot G - \sum_{i=1}^{n}e_i'P_i$, the signature $T = (s,D)$ is valid. The validity of $T$ proves a verifier that the sum of the commitments in the transaction entries results in a commitment to 0 after subtracting the transaction fees paid by the different participant, thus ensuring that no tokens are created or lost in the transaction. 

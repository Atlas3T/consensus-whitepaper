This chapter gives an overview of the cryptographic libraries and tools used to define objects (transactions) and the consensus-based protocol on Catalyst ledger that permits the update of the decentralised ledger across its peer-to-peer network. \\

Distributed Ledger Technologies (DLT) can be described as a distributed database managed by a peer-to-peer network of computers. Many forms of data, from simple text files to media files or bank accounts can be stored on a database. In centralised network, a database is typically managed by a central computer or server and some part of the database are accessible to users. On the contrary, decentralised database are replicated across the network, each computer holds a local copy of the database. The database is no longer managed by a central authority by instead by a plurality of computers (or nodes) on the network. The replication of the database across multiple nodes removes the vulnerability of single point of failure found with centralised database. Users can exchange digital data stored on a database via exchange requests, referred as transactions. To generate an exchange of data, the transactions are signed by the owners of the data being exchanged. Nodes on the network agree on the validity of the transactions issued by users via a consensus-based protocol thus authorising the transactions to take place and the database to be accordingly updated across the network. \\

DLT's relies on the generation of cryptographically secure transactions in order to remove the need for a central authority. The ownership and exchange of data is made possible via the use of asymmetric encryption where users hold public / private key pairs. The public key can be made visible to all users and is derived from the private key solely known by the user. Public keys act as users pseudonym on the network. Knowledge of the private key is necessary to successfully signed a transaction, the digital signature proving ownership of the data being transferred to another user. While it is impossible to derive the private key from a public key or digital signature with classic computers , it is easy to verify, given a public key, that a signature could only have been generated by the user in possession of the associated private key. \\

Two common asymmetric encryption techniques are RSA and Elliptic Curve (EC) cryptography. RSA's hardness relies on the difficulty of integer factorisation of large prime numbers.  EC cryptography relies on the hardness of the discrete logarithm problem. In DLTs, EC cryptography is chosen in preference to RSA due to the significantly smaller key size for the same level of security. On Catalyst, EC-based private keys have a 256-bit size which provides a 128-bit security. An equivalent RSA-based key would have a 3072-bit size.


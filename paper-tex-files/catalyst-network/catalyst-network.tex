Peer-to-peer communication allows messages (including but not limited to transactions) to be propagated across a network. Peer-to-peer networks rely upon information to be passed between nodes in an efficient and orderly manner. \\


The protocol used to propagate messages needs to be such that the large majority of nodes receive accurate messages in a timely manner. Catalyst implements a gossip protocol to propagate messages amongst peers. Gossip protocols, also known as epidemic protocols are named as such because of how they spread information. Each node propagates a message to a number of its connected peers, randomly chosen amongst nodes in the network. As nodes receive the message they propagate it to their peers. This allows the message to spread rapidly with a high level of coverage. \\

Catalyst implements a peer identification protocol. Each node that joins the network must have a unique peer identifier that describes the node’s identity. This allows users to track their connected peers as well as associate a reputation to each node, to track badly performing nodes. \\

The peer identification and gossip protocols are thoroughly documented in a technical paper currently under writing by the Atlas City engineer team. The following describes the different roles and responsibilities assumed by nodes on Catalyst Network. 

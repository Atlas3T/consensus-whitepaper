Catalyst implements a peer identification protocol. Each node that joins the network must have a peer identifier that describes the node’s identity. This allows users to track their connected peers as well as associate a reputation to each node. A peer identifier consists of the following set of parameters:

\begin{itemize}
\item \textbf{Client ID} - a 2-byte number to differentiate test networks from live network
\item \textbf{Client Version} - a 2-byte version number of code running on peer 
\item \textbf{IP} - a 16-byte IP address, IPv4 or IPv6
\item \textbf{Port} - a 2 byte port number 
\item \textbf{Public Key} - a 20-byte public key
\end{itemize}

Each peer identifier on the network should be unique. The peer identification protocol allows the Catalyst network to prevent Sybil attacks, as discussed in \ref{Sec:Reg}, and to assign a reputation to the peer identifier to track badly performing nodes. The reputation of a node is determined by how its peer identifier is formed. To maintain a good reputation, nodes need to respond correctly to requests from their peers. For example a peer can send the IP:Port number chunk of a node, and the corresponding node is required to respond with the associated public key. The node reputation decreases when providing the wrong public key or not being available while a correct response is rewarded by an increase in reputation. Multiple nodes joining on the same IP space will see their reputation decrease. \\

Peer discovery on the Catalyst network is performed using a Metropolis-Hastings Random Walk with Delayed Acceptance~\cite{hasting} (MHRWDA). This random walk is designed to cause a high cost to eclipse attacks from malicious nodes. When a new node joins the network, it connects to a random seed peer. It then performs a random walk through the peer network, recording discovered peers only after an initial burning period. This algorithm reduces the chance of revisiting previously checked nodes, as well as reducing any bias towards nodes which have a large number of peers. \\

Messages will be relayed by all nodes on the network. Catalyst implements a gossip protocol to spread messages across peers. Gossip protocols propagate messages through a network by relying on each node to communicate with their neighbours. This peer identification protocol allows nodes to check the identity of peers they are connected to, as well as their roles on the network. 
This paper gives an overview of the database architecture of the distributed ledger and the structure and different types of transactions supported on Catalyst Network. It presents the consensus protocol behind Catalyst Network, a new consensus protocol based on the collaborative work performed by the network nodes, which uses the computing resources available across the network to efficiently and securely reach a consensus on the distributed ledger state updates. This paper is organised as follows:

\begin{itemize}
\item \textbf{Chapter \ref{Cha:Tec} - Technical Specifications}: this chapter describes the cryptographic libraries and tools used in Catalyst code base, including the choice of elliptic curve, hashing algorithm and the zero-knowledge proof protocols. 

\item \textbf{Chapter \ref{Cha:NAPI} - Peer-to-peer Network}: this chapter describes the process followed by nodes joining the network and the process of peer identification. The different roles of nodes on Catalyst are explained, as well as the process to register on the network in order to perform work related to the network (and ledger database) management. 

\item \textbf{Chapter \ref{Cha:LDA} - Ledger Database Architecture}: this chapter gives an overview of the ledger database architecture as well as the different types of account stored on Catalyst. The concepts of current ledger state (CLS) and Distributed File System (DFS) for storing ledger state updates and files are introduced. 

\item \textbf{Chapter \ref{Cha:Tra} - Catalyst Transactions}: this chapter introduces the different transaction types supported on Catalyst and describes the transactions structure, including the process followed by users to generate and validate transaction signatures. 

\item \textbf{Chapter \ref{Cha:CM} - Catalyst Consensus Mechanism}: this chapter presents the new consensus mechanism implemented on Catalyst. 

%\item \textbf{Chapter \ref{Cha:Sha} - Catalyst Scalability}: this chapter describes how sharding techniques are implemented to Catalyst consensus mechanism to maintain the network performance at very large scale. 

%\item \textbf{Chapter \ref{Cha:Ben} - Benchmarking}: this chapter presents results of expected performances of the network for different scaling scenarios.

\item \textbf{Chapter \ref{Cha:Sec} - Security Considerations}: this chapter discusses security considerations with regards to the signature scheme and the consensus-based protocol on Catalyst. 

\end{itemize}
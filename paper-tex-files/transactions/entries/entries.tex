Transaction entries are used on the Catalyst ledger to represent the transfer of tokens from or to the account referenced in the entries. This generally takes the form of debit or credit of an account. We use the term entry to replace the traditional input and output, as on Catalyst there is no differentiation as to how debit or credit of an account is formed. Whether spending or receiving tokens, a user must sign their transaction entry and a transaction is complete if and only if all transaction entries have been signed.\\

 A transaction entry typically consists of two components: 
 \begin{itemize}
\item A public key, from which the address of an account stored on the ledger is derived.
\item An amount component that can be a number or a more complex structure and represents the amount of tokens spent from or transferred to the address associated to the public key.
%\item and a fee component that is a number representing the amount of tokens paid as transaction fees. 
\end{itemize}
On Catalyst, the number  or amount of tokens included in a transaction entry can be positive (when receiving) or negative (when spending). This  choice allows for a) keeping a simple transaction entry structure (there is no need for an extra field to specify the type of transfer embedded in an entry) and b), in the case of confidential asset transfer, an improved anonymity as an observer will be unable to differentiate between a sender and a recipient in a token transfer. \\

The public key $Q_i$ in a transaction entry $E_i$ is always a 32-byte element from which one address $A_i$ stored on the ledger can be derived. \\

The amount component of an entry however differs depending on the nature of the token exchange:

\begin{itemize}
\item For non-confidential asset transfer, it is a 8-byte (positive or negative) number $v^t_i$ that represents the number of tokens spent from or transferred to the address $A_i$, communicated in clear text.
\item For confidential asset transfer, it is made of two elements: 
\begin{itemize}
\item A 32-byte Pedersen commitment $C_{i}^{t}$ that represents the commitment of tokens spent from or transferred to the address $A_i$ 
\item A range proof $\Pi_i(C_{i}')$ (as discussed in section \ref{Sec:ZKP}) that proves that the token balance of $A_i$ remains within an acceptable range of value (typically greater than 0 and smaller than a threshold $M$ of number of tokens) after the transaction has taken place.
\end{itemize}
 The construction of these elements is discussed below. 
\end{itemize}

Assume that the balance  $v_i$ of the account $A_i$ is initially represented on the ledger by the PC: 
\begin{equation}
C_{i}=(v_i~mod~l_H)H+ (b_i~mod~l_G)G
\end{equation}
Where $b_i$ is a blinding factor chosen by the account holder.  The latter wishes to transfer or receive $a_i$ tokens. To obfuscate the amount of tokens transferred in a transaction entry $E_i$, the account holder creates a PC: 
\begin{equation}
C_i^{t} = (v_i^{t}~mod~l_H)H+ (b_i^{t}~mod~l_G)G
\end{equation}
Where $v_i^{t} = a_i$ if receiving the tokens ($a_i > 0$) and $v_i^{t} = l_H+a_i$ if spending the tokens ($a_i<0$).\\

Given the properties of the Elliptic curve, $l_H$ is much greater than $M$ in the second case. As a result, it would not be possible to construct a valid range proof for $v_i^{t}$. $U_i$ thus creates a second PC:
\begin{equation} 
C_{i}'=(v_i'~mod~l_H)H+ (b_i'~mod~l_G)G
\end{equation}
Defined as:
\begin{center}
\[
  C_{i}' = C_i + C_i^{t}\begin{cases}
               v_i' = v_i + v_i^t\\
               b_i' = b_i + b_i^t
            \end{cases}
\]
\end{center}
It represents the commitment of the account balance after the transaction has taken place.  If $a_i<0$, $v_i'H = [v_i + (l_H+a_i)mod~l_H]H = [v_i + a_i]H$. It is only possible to generate a valid range proof associated to $C_{i}'$ if $v_i > a_i$. Note that $v_i$ is necessarily smaller than $M$ as the balance of the account would have been determined by a previous transaction entry, itself including a range proof that $v_i~\in~[0,M]$. As discussed in section~\ref{Sec:ZKP}, a range proof generated using the bulletproof protocol amounts to 672 Bytes.  \\

Table~\ref{tab:TrSt} summarises the different components of a transaction and their respective size for the two types of transfer aforementioned. \\

\begin{table}[htbp]
\centering
\begin{tabular}{ | l | l | l | l | }
    \hline    
    \multicolumn{3}{| l | }{\textbf{Transaction component}} &   \textbf{Size} \\ \hline \hline
     \multicolumn{3}{| l | }{Type}  & 4 Bytes \\ \hline
     \multicolumn{3}{| l | }{Timestamp}  & 4 Bytes \\ \hline

       \multirow{ 4}{*}{Entries ($n>1$)} & \multirow{ 2}{*}{non-confidential entry} & 32-byte public key &  \multirow{ 2}{*}{$n \cdot 40$ Bytes} \\
     \cline{3-3}
    	& & 8-byte amount &\\  
     \cline{2-4}

                   &   \multirow{ 2}{*}{or confidential entry} & 32-byte public key &  \multirow{ 2}{*}{$n \cdot 736$ Bytes} \\
     \cline{3-3}
   	&   & 32-byte PC &\\ 
     \cline{3-3}
	& & 672-byte range proof &\\ \hline
   \multicolumn{3}{| l | }{Aggregated Signature} & 64 Bytes\\  \hline
 \multicolumn{3}{| l | }{Transaction fees} & 8 Bytes \\  \hline   
 \multicolumn{3}{| l | }{Locking Time}  & 4 Bytes \\  \hline   
    \end{tabular} \\ 
\caption{Structure of confidential and non-confidential transactions on Catalyst and size per transaction component.}
\label{tab:TrSt}
\end{table}

Each entry in a transaction needs to be signed to authorise the transfer of tokens from or to the address included in said entry. This can be achieved through the use of an aggregated signature scheme as described in section \ref{Sec:Sig}. \\

Another type of transaction entry is considered on Catalyst, that is a stand-alone entry. It is not included in a transaction but is added to the ledger state update generated by the producers during the ledger cycle and includes the reward allocated to a specific producer for its contribution in producing a valid update of the ledger state. Such entry, called \textit{ledger compensation entry} (or simply compensation entry) is very similar to a non-confidential entry. It includes a 8-byte amount, that is however always positive, and a 32-byte public key from which the address of an account stored on the ledger is derived. However, unlike transaction entry, a compensation entry need not be signed to authorise the transfer of tokens to the account address specified in said entry. 


In traditional blockchains (such as Bitcoin) a transaction is composed of a set of inputs and outputs. An input refers to the output of a transaction stored on a valid block of the blockchain, effectively spending that output (also referred as UTXO). In broad terms, an input thus spends tokens, while an output receives some. The output is locked and can later be spent in an input of a future transaction. On the Catalyst ledger, we opt for a new terminology, defining as \textit{transaction entry} a transaction component that spends or receives tokens.\\

A transaction object on Catalyst is made of the following components:

\begin{itemize}
\item A transaction type specifying the type of exchange embedded in the transaction entries (non-confidential or confidential asset transfer, data storage request and retrieve, smart contracts-related token and/or data transfer).      
\item A timestamp corresponding to the point in time the transaction is complete and ready to be broadcast on the network.
\item A set of $n$ transaction entries $\{E_i\}_{i=1,..,n}$. Transaction entries are specific to the nature of the token and data exchange. These are described in \ref{Sec:TEnt}.
\item An aggregated signature $T$ proving ownership of the set of accounts called in the transaction entries.
\item A locking time corresponding to a point in time after which the transaction can be processed by a worker pool.
\item the transaction fees paid the transaction participants
\item A data field that can contain up to $x$ Bytes of data transferred in data storage or smart contract-related transactions. 
\end{itemize}

Any valid transaction must contain a type, a timestamp and locking time (the later is set to 0 is there is no waiting period prior to processing a token exchange embedded in the transaction), a list of transaction entries and an associated signature. Additional fields can be included to the transaction object, depending on the nature of the token exchange. 
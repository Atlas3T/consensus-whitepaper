Proof-of-Work (PoW) and derivate algorithms are commonly used to manage blockchain and distributed ledger in a distributed manner. Consensus-based protocols based on such algorithms rely on a plurality of nodes, called miners, competing to generate at fairly regular interval of time a valid block of transactions to append to the blockchain. Part of the competition consists in solving a cryptographic puzzle that ensures the validity of the content of a block. \\

However, this competition amongst nodes wastes a tremendous amount of energy as all miner nodes expend computational power to solve the same problem, yet only the work performed by one node is used to update the blockchain. The energy consumption per year for Ethereum and Bitcoin combined is 66.6 TWh per year which is comparable to yearly energy consumption of Switzerland (61.6 TWh per year) \cite{electric}\footnote{This energy consumption allows approximately 445 million transactions for Bitcoin and Ethereum combined per year \cite{BitTxpD}\cite{EthTxpD}, compared to Switzerland where 820 million debit card transactions are processed per year \cite{swis} for an estimated energy consumption of 0.001358 TWh.}. It is clear that this is not sustainable or environmentally friendly. Moreover, as the difficulty associated with the cryptographic puzzle increases over time, miners are forced to invest in more computer resources to have a chance of earning miner rewards. Such consensus protocols have a clear negative environmental impact and indicate counteractive economic implications with high risk of mining centralisation. \\

This section presents a new consensus-based protocol that can be applied to a peer-to-peer network in order to manage a distributed ledger in a fair and secure manner without wasting unnecessary amount of energy. 
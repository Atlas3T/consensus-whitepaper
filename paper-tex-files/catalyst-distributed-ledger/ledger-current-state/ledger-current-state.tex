The ledger state thus encompasses different partitions, each of which keeps the balance of accounts of a specific type up-to-date. The current ledger state (CLS) lists the accounts balance at the present time, allowing anyone to access (and comprehend in the case of non-confidential accounts) the available balance in tokens of an account. \\

When users on the network wish to transfer tokens to other users, they issue transactions that are broadcast to the network. The structure of these transactions are discussed in section~\ref{Sec:TStru}. The transactions are collected by nodes assigned to the management of the ledger database (as detailed in section~\ref{Sec:PRT}) and used to generate a ledger state update. A ledger state update is a cryptographically secure structured data object that allow users to update their local copy of the ledger. The production of valid ledger state updates in a trust-less environment is discussed in section~\ref{Sec:Dem}. \\

The ledger state update consists of a summary of the token transfers embedded in the transactions broadcast by the network users. Transactions broadcast during a ledger cycle are collected by nodes who then use these to generate a ledger state update during the next ledger cycle. In layman's terms, the ledger state update can be viewed as a structured database with a series of row, each row having two components: a public key referring to the address of an account stored on the ledger and an amount (positive or negative) that represents a token transfer.\\ 

Let's assume for instance that Alice wants to transfer 5 KAT tokens to Bob. The transfer form Alice's account to Bob's account would be represented by a transaction with two entries. The ledger state update including this transaction would comprise two rows: one row with Alice's account address (or the public key used to derived the account address) and a negative amount $-5~KAT$ and one row with Bob's address and a positive amount $5~KAT$ (transaction fees, discussed in section~\ref{Sec:TStru}, are ignored here). Once a user receives a valid ledger state update, the former can use the latter to update their local copy of the ledger: Alice account is debited of 5 tokens while Bob's account is credited of 5 tokens. Note that the ledger state update produced for one ledger cycle only includes balance changes of accounts called in the transactions broadcast on the network during the precedent ledger cycle. This allows for a compact ledger state update as there may be many more accounts stored on the ledger that are not used during a ledger cycle.\\

Transactions in the context of DLT refer to data objects created and cryptographically signed by users and propagated as messages on the peer-to-peer network. A transaction \textit{a la} Bitcoin typically includes:

\begin{itemize}
\item a set of inputs where each input comprises the details of the account or digital address being debited, the (positive) amount associated to that address and the signature of the account owner, proving the legitimacy or ownership of the tokens as well as the valid balance of the debited account. 
\item a set of outputs where each output comprises the details of the account being credited. Rather than a signature, a locking program is attached to the output, that effectively locks the tokens sent to this output using the public key of the recipient (the user holding the account being credited).
\end{itemize}

A digital signature associated to a transaction input is a mathematical scheme that allows the owner of the associated private key to prove that they have authorised the spending of the funds locked in the output of a transaction stored on the ledger. A valid signature further guarantees the non-repudiation of the message (the sender cannot deny having sent the message) and the message integrity (the latter has not and cannot be tampered with). \\

On the Bitcoin blockchain, valid blocks of transactions get appended to the blockchain in such a way that any new block is cryptographically sealed and linked to the last block appended to the blockchain. A block contains a set of transactions that transfer digital assets from a set of digital addresses to another set, as well as an extra transaction, called a \textit{coinbase} transaction, that rewards the miner who successfully produced that block with new digital coins. Each transaction input contained in a valid block (except the coinbase transaction) refers to the output of a transaction stored on a previous block (\textit{a.k.a} an unspent transaction output – UTXO). It can actually be viewed as the second state of that output. First, the output is unspent, locked and stored on a valid block. Secondly, the output is used as input in a new transaction and unlocked by the owner of the unlocking key. Eventually, an old block in the blockchain will solely contain spent transaction outputs usable as inputs in transactions stored in later blocks. As such the old block becomes obsolete as it no longer holds any spendable tokens. \\

The Catalyst ledger operates differently in the sense that it does not store UTXOs. The ledger state comprises digital accounts of which the balance changes over time as transactions debiting or crediting these accounts are validated on Catalyst network. As detailed in sections \ref{Sec:TStru} and \ref{Sec:Dem}, the removal of UTXOs is made possible via the combination of a novel consensus-based protocol and a new transaction structure such that any token transfer embedded in a transaction (whether spending or receiving tokens) is signed and thus authorised by the relevant parties involved in said transfer. User nodes need not access old ledger state updates to be able to transfer tokens from their account stored on Catalyst ledger. They only need a local copy of the current ledger state.\\

Once a ledger state update is generated by a pool of producer nodes, it is stored on DFS and can be accessed by any node to update their local copy of the current ledger state. DFS is built upon the IPFS protocol~\cite{ipfs} and is used to store files as well as historical ledger state updates. This removes the burden on user nodes to maintain the full history of the ledger database while allowing for fast retrieval of files as well as old ledger state updates. DFS is maintained by all nodes on the network. However, DFS is made of a multitude of compartments and each node needn’t hold all compartments. The design of a ledger compartment dedicated to the storage of files and historical ledger state updates is an approach taken to prevent the bloating of the ledger and allow the network to support services at scale. Indeed, this approach allows Catalyst ledger to remain both lean and cryptographically secure.

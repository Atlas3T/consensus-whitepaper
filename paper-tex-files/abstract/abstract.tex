Distributed Ledger Technology (DLT) is undoubtedly one of the most disruptive new technologies to have emerged over the past decade. When the paper~\cite{nakamoto} was published in 2008 by an unknown author under the famous pseudonym Satoshi Nakamoto, the world was shaken by a global economical crisis that considerably deteriorated our trust in central authorities. The new protocol embedded in this paper proposed a revolutionary approach to conduct transfers of digital assets, without the need of a third party, thus opening the door to a new economical approach with greater transparency, privacy and prosperity for all. Bitcoin was the first successful public blockchain to demonstrate the potential for this technology to be used as a decentralised yet trusted store of value. Building on this early success, next generation blockchains such as Ethereum and Neo demonstrated the potential for blockchain platforms to provide decentralised computing services, enabling more complex applications and reaching more markets than straight forward storage of value. Subsequent blockchains and distributed ledgers established use-cases in many other areas notably through the use of Internet of Things devices and machine learning techniques~\cite{govuk}.\\

As Bitcoin and other projects grew in popularity it became apparent that real challenges awaited this new technology. Indeed, the early systems were not built to meet the demand for services at scales comparable to those of cloud services. The main challenge of blockchain is to solve the so-called blockchain trilemma, building a system that can process a high throughput of transactions while ensuring the system integrity and accessibility to all. Additional concerns arise notably around environmental impact and the risk of power centralisation that would inevitably lead to the level of wealth disparity we observe in a world governed by centralised systems. A plethora of projects started with in mind to tackle these challenges (and more).\\

Building a blockchain or distributed ledger is a complex task and for that reason most existing projects are clones, also known as forks, made from a small number of original blockchains. This allows organisations to benefit from already developed blockchains while modifying the elements relevant to their field. The problem with such an approach is that it restricts truly original thinking about wider technological issues such as how a network can scale or operate in environments for which the original Bitcoin blockchain was never designed. As a result of forking from the past, the fundamental issues restricting present blockchain technologies such as scale, privacy, performance and interoperability remain as much of a challenge today as when these early blockchains were first developed~\cite{obst}.\\

Atlas City took a very different approach when designing a new core protocol ledger and accompanying distributed computing capability, starting from a set of operational requirements and developing a cohesive system that delivers to those requirements. The code base developed by Atlas City researchers and engineers is original and will be made available as an open-source software. To solve the fundamental issues inhibiting the growth of distributed ledger-based computing, engineers and researchers at Atlas City were and are encouraged to ask and rethink fundamental questions about the new distributed operating system they envisage.\\

Learning from popular and new blockchains and distributed ledgers as well as the wider IT industry, Atlas City developed Catalyst, a full-stack distributed network  built to fulfil the real-world potential of DLT, to enable the next generation of distributed computing applications and business models. This paper presents the consensus protocol of Catalyst Network. 

\subsection{Rogue Key Attack}

When Schnorr signatures used to generate aggregated signature of a transaction are vulnerable to an attack known as Rogue Key attack. Rogue Key attacks performed by a malicious entity consists of generating an aggregated signature in such a way that they posses the public/private key pair for that signature. In the Schnorr signature scheme, the public key of participants are aggregated and the sum represent the public key associated to the signature. Assume that an honest participant use its public key $Q_a$ in the transaction and a malicious participant possesses $Q_b$. By sending the public key $Q_m = Q_b - Q_a$ to the honest participant, the malicious entity have access to the transaction as they will hold the private key for $Q_b$. This is because when the keys are aggregated i.e. $Q_m + Q_b$ the aggregated signature would be $Q_b$, for which the malicious user holds the private key (the honest user would not). For an aggregated public key, there should be no user that has a private key equivalent as it should be used to create a signature that can be verified that all users in the transaction participated. \\ 	

The aggregation of public keys used in Catalyst which is based on Mu-Sig \cite{musig} signature scheme that is not vulnerable to this form of attack. Mu-Sig is protected from this form of attack as the scheme does not require a user to demonstrate each public key, only the sum of all the public keys. By not verifying individual public keys, a key rogue attack is not possible. Only one public key is needed for the verification (the aggregated key) for which there will not be an equivalent private key.

\subsection{Quantum Attack}

Quantum computers pose a very real threat to the encryption techniques used in blockchains in the medium to long term \cite{agarwal} \cite{kearney}. The threat is through the use of Shor's algorithm. A quantum attacker using Shor's algorithm on a quantum computer can gain an exponential speed-up in solving the discrete logarithmic problem. The assumption of security the discrete logarithmic functions the the primary basis as to which all elliptic curve cryptography is based. This means that even the schema demonstrated here will be vulnerable to attack. The use of aggregated signatures would provide some resistance, however this resistance would be negligible. \\

It must be impressed that this is not an issue for the near term and thereby, these schema are highly secure and efficient currently. The most efficient algorithm for classical computers to solve the discrete logarithm problem is the Pollard's rho\cite{pollard}, this does not run in polynomial time. While Catalyst is not currently resistant to quantum attack, this is a challenge that will be faced by all major distributed ledgers over time. 
The individual technical components underpinning Distributed Ledger and Blockchain Technologies have existed for decades. As the $1^{st}$ generation blockchain, Bitcoin managed to recombine these previously established elements in a unique fashion so as to instil and enable trust in a trust-less system, thus achieving decentralisation and eliminating the need for a centralised authority. Whilst completely revolutionary at the time, the implementation and expansion of this new approach uncovered limitations and hurdles to both expanded use and ultimately mainstream adoption. $2^{nd}$ generation DLTs and blockchains built and improved upon this original foundation but fall short of resolving all associated issues. \\
 
Atlas City has developed a distributed operating system, Catalyst, to solve the issues of previous DLTs and blockchains, improving upon those which came before, resolving such challenges and enabling an equitable and proportionate compensation to participants on the network. Catalyst was designed around the notion that a democratic and ethical network can exist which is secure, decentralised, scalable and private.  \\
 
Catalyst code base does not fork from a previously existing projects and includes original and innovating work, including a new collaborative and environment-friendly consensus-based protocol, the possibility to process both confidential and non-confidential transactions as well as smart contracts, an efficient peer-to-peer communication layer and a multi-level data architecture for a lean ledger database storing a variety of data.\\
 
This paper gives an overview of Catalyst core protocol. Another paper, currently under preparation, presents network performances measurement. 


On the Catalyst ledger, Elliptic Curve (EC) cryptography is used to sign messages and generate proofs of knowledge of information amongst users without having to reveal any information. \\

Throughout this paper, EC points are used for the creation of public keys and Pedersen Commitments (PC). EC cryptography techniques are also used to generate and verify the signature of transaction encompassing the transfer of KAT tokens (see section~\ref{Sec:Tok}) from or to accounts locked by the private keys of Catalyst users. Finally, EC cryptography techniques are used for the generation and verification of range proof that any number (\textit{i.e.} an amount of tokens) hidden in a PC can be provably shown to lie within a range of acceptable values.\\

There exists many types of EC, some of which are part of NIST curves~\cite{Hankerson2011}. While the NIST curves, such as the secp256r1 curve, are advertised as being chosen verifiably at random, there is little explanation for the seeds used to generate these. By contrast, the process used to pick non-NIST curves, such as the twisted Edwards Curve25519 used in Monero project~\cite{monero}, is fully documented and rigid enough that independent verifications can and have been done. This is widely seen as a security advantage, since it prevents the generating party from maliciously manipulating the curve parameters~\cite{NSA}. Moreover, EC such as Curve25519 are designed to facilitate the production of high-performance constant-time implementations. \\

On the Catalyst ledger we opt for the twisted Edwards Curve25519 \cite{ed25519}, that is a birational equivalent of the Montgomery curve Curve25519. It is defined over the prime field $\mathbb{F}_{p}$ where $p = 2^{255} - 19$, by the following equation:
\begin{center}
$-x^2+y^2=1-\frac{121665}{121666}x^2y^2$
\end{center}
The order of Curve25519 can be expressed as $N=2^cl$ with $c$ a positive integer and $l$ a 253-bit prime number. $N$ is a 76-digit number equal to:
\begin{center}
\begin{footnotesize}
$N=2^3\cdot 7237005577332262213973186563042994240857116359379907606001950938285454250989$
\end{footnotesize}
\end{center}

Elements in $\mathbb{F}_{p}$ are 255-bit integers and can thus be represented in 32 bytes with the most significant bit set to 0. An EC point on the twisted Edwards Curve25519 would therefore be represented with 64 bytes. But given point compression techniques described in~\cite{monero}, it is possible to reduce an EC point to a 32-byte data where 255 bits represents the $x$ coordinate of the point and the last bit indicates the $y$ coordinate. \\

In general terms, an EC of the form $y^2 = x^3 + ax + b$ is defined over a prime field $\mathbb{F}_p$ where $p$ determines the maximum values of $x$ and $y$, the two coordinates of an elliptic curve point. The elliptic curve has a cyclic group of $n$ points. A EC generator, $G$ for instance, is an EC point itself generating a cyclic subgroup of order $l_G \leq n$. This subgroup is composed of the set of points: $\{0G, 1G, 2G,...,(l_G-1)G\}$ with $0G = l_GG$ known as the point at infinity. This subgroup is defined by the relation $xG = (x ~mod~l_G)G$. The order of the subgroup $l_G$ is a divisor of $n$. For instance, if $n = 100$, then $l_G$ can take a value in $\{2, 5, 10, 25, 50\}$. Note that $n$ can not be a prime number. If $l_G = n$, the subgroup of $G$ includes all the points of the EC.
